\documentclass{article}
\usepackage{amsmath}
\usepackage{mathtools, nccmath}
\usepackage{amssymb, amsthm, mathrsfs}
\begin{document}

According to the decompilation of the Ciso Vigenere hash algorithm, when the password length is less than 16 the idea behind Ciso Vigenere hash algorithm is: \\
Let p be the password that the user types. \\
Let hp be the hardcoded password in the code of Packet Tracer. \\
Let lp be the length of the user input password. \\
Let h be the hash value obtained from the custom algorithm. \\
So that:

$$
\begin{flushleft}
\begin{multline}
\[

\forall h \forall lp \forall hp [(hp = (d, s, f, d, ;, k, f, o, A, ,, ., i, y, e, w, r, k, l, d, J, K, D, H, S, U, B, s, g, v, c, a, 6, 9, 8, 3, 4, n, c, x , v), \\
0 \textless lp \textless 16, \\
h_{0} = 0, \\
h_{1} = 8, \\
h = \Sigma_{i=2}^{lp}
\begin{cases}
    ((p_i \oplus hp_{8 + i}) \ggg 4) + 0x30,                                   & \text{if } (p_{i} \oplus hp_{i+8} \land 0xfffffff0 < 0xa0)        \text{ and if } i \equiv 0 \pmod 2 \\
    ((p_i \oplus hp_{8 + i}) \ggg 4) + 0x37,                                   & \text{if } (p_{i} \oplus hp_{i+8} \land 0xfffffff0 \geq 0xa0)     \text{ and if } i \equiv 0 \pmod 2 \\
    ((p_i \oplus hp_{8 + i}) \land 0xf) + 0x30,                                & \text{if } (p_{i} \oplus hp_{i+8} \land 0xf < 0x0a)               \text{ and if } i \equiv 1 \pmod 2 \\
    ((p_i \oplus hp_{8 + i}) \land 0xf) + 0x37,                                & \text{if } (p_{i} \oplus hp_{i+8} \land 0xf \geq 0x0a)            \text{ and if } i \equiv 1 \pmod 2
\end{cases} \\
) \implies \nexists p[p = \mathbf{rev}(h)] \\

\]
\end{multline}
\end{flushleft}
$$

So let's split each sub steps of the algorithm. In this wayt, we could start prooving that if $ P \implies Q $ and if $ Q \implies R $ then $ P \implies R $

So for any P so that:

$$
\begin{flushleft}
\begin{multline}
\[

h = \Sigma_{i=2}^{lp}
\begin{cases}
    (p_{i} \oplus hp_{i+8} \land 0xfffffff0 < 0xa0)        \text{ if } i \equiv 0 \pmod 2 \\
    (p_{i} \oplus hp_{i+8} \land 0xfffffff0 \geq 0xa0)     \text{ if } i \equiv 0 \pmod 2 \\
    (p_{i} \oplus hp_{i+8} \land 0xf < 0x0a)               \text{ if } i \equiv 1 \pmod 2 \\
    (p_{i} \oplus hp_{i+8} \land 0xf \geq 0x0a)            \text{ if } i \equiv 1 \pmod 2
\end{cases} \\
) \implies \nexists p[p = \mathbf{rev}(h)] \\

\]
\end{multline}
\end{flushleft}
$$


So for any Q so that:

$$
\begin{flushleft}
\begin{multline}
\[

h = \Sigma_{i=2}^{lp}
\begin{cases}
    (p_{i} \oplus hp_{i+8} \land 0xfffffff0 < 0xa0),                 \text{ if } i \equiv 0 \pmod 2 \\
    (p_{i} \oplus hp_{i+8} \land 0xfffffff0 \geq 0xa0)               \text{ if } i \equiv 0 \pmod 2 \\
    (p_{i} \oplus hp_{i+8} \land 0xf < 0x0a),                        \text{ if } i \equiv 1 \pmod 2 \\
    (p_{i} \oplus hp_{i+8} \land 0xf \geq 0x0a),                     \text{ if } i \equiv 1 \pmod 2
\end{cases} \\
) \implies \forall p[p = \mathbf{rev}(h)] \\
\]
\end{multline}
\end{flushleft}\\
$$


Let's start by prooving 

$$
\begin{flushleft}
\begin{multline}
\[

\forall h \forall lp \forall hp [(hp = (d, s, f, d, ;, k, f, o, A, ,, ., i, y, e, w, r, k, l, d, J, K, D, H, S, U, B, s, g, v, c, a, 6, 9, 8, 3, 4, n, c, x , v), \\
0 \textless lp \textless 16, \\
h_{0} = 0, \\
h_{1} = 8, \\
h = \Sigma_{i=2}^{lp}
\begin{cases}
    ((p_i \oplus hp_{8 + i}) \ggg 4) + 0x30,                                  & \text{if } (h_{i} \oplus hp_{i+8} \land 0xfffffff0 < 0xa0)        \text{ and if } i \equiv 0 \pmod 2 \\
    ((p_i \oplus hp_{8 + i}) \ggg 4) + 0x37,                                  & \text{if } (h_{i} \oplus hp_{i+8} \land 0xfffffff0 \geq 0xa0)     \text{ and if } i \equiv 0 \pmod 2 \\
    ((p_i \oplus hp_{8 + i}) \land 0xf) + 0x30,                               & \text{if } (h_{i} \oplus hp_{i+8} \land 0xf < 0x0a)               \text{ and if } i \equiv 1 \pmod 2 \\
    ((p_i \oplus hp_{8 + i}) \land 0xf) + 0x37,                               & \text{if } (h_{i} \oplus hp_{i+8} \land 0xf \geq 0x0a)            \text{ and if } i \equiv 1 \pmod 2
\end{cases} \\
) \implies \nexists p[p = \mathbf{rev}(h)] \\

\]
\end{multline}
\end{flushleft}
$$

## I/ substraction to reverse the addition

$\forall x [(x = y + z) \implies (y = e \minus z)]$ then it follow that as the previous part of the function contains: $ h = x + 0x30 $, then $ h - 0x30 = x $ so $ \exists rev(h)[rev(H(p)) =p - 0x30] \\$ 

## II/ exclusive or

According to the Karnaught table at: https://fr.wikipedia.org/wiki/Table_de_v%C3%A9rit%C3%A9#Disjonction_exclusive, $ \forall x [(x \oplus x) \implies (x = 0)] $.
$$
\\
$$
Then as $ xlat \oplus xlat = 0 $, and as $ p \oplus 0 = p $, we know that the original password $p = xlat \oplus h $. $\\$

## III/ rotating 4 first to 4 last bits

$ \forall x [(x \ggg y) \implies (x \lll y = x)] $. \\

Then as $z = (x \ggg y) = (x \lll y) $, we know that the original password $ p = H(p) \lll 4 $.
$$
\\
\\
$$

## IV/ unmasking different signatures (recurrent marks) in the hash

In the previous chapter one `I/ substraction to reverse the addition`, we told we can reverse the previous addition. We still need to guess which addition/substraction has been done previously.

As both addition values are made depending of: \\
if $ (password_left & 0xf0 < 0xa0) \implies (password_left & 0xf0 + 0x30) $ or else $ (password_left & 0xf0 > 0xa0) \implies (password_left & 0xf0 + 0x37)  \\ $
if $ (password_right & 0x0f < 0x0a) \implies (password_right & 0x0f + 0x30)$ or else $ (password_right & 0x0f > 0xa0) \implies (password_right & 0x0f + 0x37) \\ $

So if the out has the 4 four bits value so that: 
$ x \in { x | (0xf0 & x) \leq 0xa0 } \implies y = x + 0x30 $ $\\$

So if the out has the 4 four bits value so that: 
$ x \in { x | (0xf0 & x) > 0xa0 } \implies y = x + 0x37 $ $\\$

So if the out has the 4 four first bits value so that:
$ x \in { x | (0x0f & x) \leq 0x0a } \implies y = x + 0x30 $ $\\$

So if the out has the 4 four first bits value so that:
$ x \in { x | (0x0f & x) > 0x0a } \implies y = x + 0x37 $ $\\$

first byte: \\
  $ 0xa0 < 0xf0 + 0x30 < y \\ $ 
  then:\\
  -1: $ x \in { x | 0xa0 < x } \implies [y \in { y | 0xc7 < y < 0xa7 }] \\$
  -2: $ x \in { x | x < 0xa0 } \implies [y \in { y | 0xc0 < y < }] \\$

second byte:
  $ 0xa0 < 0x0f + 0x30 < y \\ $ 
  -1: $ x \in { x | x < 0x0a } \implies [y \in { y | 0x3a < y }] \\$
  -2: $ x \in { x | 0x0a < x } \implies [y \in { y | y < 0x4a }] \\$


Then for both of any subnumber:

$ \forall y = H(x), x \in { x | x \leq 0xa } \implies y = x + 0x30$ $\\$
$ \forall y = H(x), x \in { x | x > 0xa } \implies y = x + 0x37 $ $\\$

It follows:

$ \forall y = H(x), y \in { y | 0 < y \leq 0x0a + 0x30 } \implies x = y - 0x30 $ then $ 0 < x < 0x0a $ $\\$
$ \forall y = H(x), y \in { y | 0 < y \leq 0x0a + 0x37 } \implies x = y - 0x30 $ then $ 0xa < x < 0x13 $ $\\$

# V /communtativity:

Addition, substraction and $ \oplus $ are commutative. \\ \\

# VI / proof

Then we have already proven each piece of the theorem so that:

$hp = (d, s, f, d, ;, k, f, o, A, ,, ., i, y, e, w, r, k, l, d, J, K, D, H, S, U, B, s, g, v, c, a, 6, 9, 8, 3, 4, n, c, x , v) \implies (\forall x \in hp[0 \geq x 0 \geq 256 \implies x \in hp]) $

then:

$$
\begin{flushleft}
\begin{multline}

Let p be the password that the user types. \\
Let hp be the hardcoded password in the code of Packet Tracer. \\
Let lp be the length of the user input password. \\
Let h be the hash value obtained from the custom algorithm. \\
So that:

\[

\forall h \forall lp \forall hp [(hp \in N \land 0 \geq hp, \\
0 \textless lp \textless 16, \\
h_{0} = 0, \\
h_{1} = 8, \\
h = \Sigma_{i=2}^{lp}
\begin{cases}
    (((p_{i} \oplus hp_{i+8}) \lll 4) - 0x30),                               & \text{if } h_i < 0xa0             \text{ and if } i \equiv 0 \pmod 2 \\
    (((p_{i} \oplus hp_{i+8}) \lll 4) - 0x37),                               & \text{if } h_i \geq 0xa0          \text{ and if } i \equiv 0 \pmod 2 \\
    (((p_{i} \oplus hp_{i+8}) \land 0xffffffff0) - 0x30),                    & \text{if } h_i < 0x0a             \text{ and if } i \equiv 1 \pmod 2 \\
    (((p_{i} \oplus hp_{i+8}) \land 0xffffffff0) - 0x37),                    & \text{if } h_i \geq 0x0a          \text{ and if } i \equiv 1 \pmod 2
\end{cases} \\
) \implies \forall p[p = \mathbf{rev}(h)] \\

\]
\end{multline}
\end{flushleft}\\
$$
\end{document}